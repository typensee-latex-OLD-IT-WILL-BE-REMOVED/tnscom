%\documentclass[12pt,a4paper]{scrartcl}
\documentclass[12pt,a4paper]{article}

\makeatletter % Technical doc - START

\usepackage[utf8]{inputenc}
\usepackage[T1]{fontenc}
\usepackage{ucs}

\usepackage[french]{babel,varioref}

\usepackage[top=2cm, bottom=2cm, left=1.5cm, right=1.5cm]{geometry}
\usepackage{enumitem}

\usepackage{multicol}

\usepackage{makecell}

\usepackage{color}
\usepackage{hyperref}
\hypersetup{
    colorlinks,
    citecolor=black,
    filecolor=black,
    linkcolor=black,
    urlcolor=black
}

\usepackage{amsthm}

\usepackage{tcolorbox}
\tcbuselibrary{listingsutf8}

\usepackage{ifplatform}

\usepackage{ifthen}

\usepackage{cbdevtool}



% MISC

\newtcblisting{latexex}{%
	sharp corners,%
	left=1mm, right=1mm,%
	bottom=1mm, top=1mm,%
	colupper=red!75!blue,%
	listing side text
}

\newtcblisting{latexex-flat}{%
	sharp corners,%
	left=1mm, right=1mm,%
	bottom=1mm, top=1mm,%
	colupper=red!75!blue,%
}

\newtcblisting{latexex-alone}{%
	sharp corners,%
	left=1mm, right=1mm,%
	bottom=1mm, top=1mm,%
	colupper=red!75!blue,%
	listing only
}


\newcommand\env[1]{\texttt{#1}}
\newcommand\macro[1]{\env{\textbackslash{}#1}}



\setlength{\parindent}{0cm}
\setlist{noitemsep}

\theoremstyle{definition}
\newtheorem*{remark}{Remarque}

\usepackage[raggedright]{titlesec}

\titleformat{\paragraph}[hang]{\normalfont\normalsize\bfseries}{\theparagraph}{1em}{}
\titlespacing*{\paragraph}{0pt}{3.25ex plus 1ex minus .2ex}{0.5em}


\newcommand\separation{
	\medskip
	\hfill\rule{0.5\textwidth}{0.75pt}\hfill
	\medskip
}


\newcommand\extraspace{
	\vspace{0.25em}
}


\newcommand\whyprefix[2]{%
	\textbf{\prefix{#1}}-#2%
}

\newcommand\mwhyprefix[2]{%
	\texttt{#1 = #1-#2}%
}

\newcommand\prefix[1]{%
	\texttt{#1}%
}


\newcommand\inenglish{\@ifstar{\@inenglish@star}{\@inenglish@no@star}}

\newcommand\@inenglish@star[1]{%
	\emph{\og #1 \fg}%
}

\newcommand\@inenglish@no@star[1]{%
	\@inenglish@star{#1} en anglais%
}


\newcommand\ascii{\texttt{ASCII}}


% Example
\newcounter{paraexample}[subsubsection]

\newcommand\@newexample@abstract[2]{%
	\paragraph{%
		#1%
		\if\relax\detokenize{#2}\relax\else {} -- #2\fi%
	}%
}



\newcommand\newparaexample{\@ifstar{\@newparaexample@star}{\@newparaexample@no@star}}

\newcommand\@newparaexample@no@star[1]{%
	\refstepcounter{paraexample}%
	\@newexample@abstract{Exemple \theparaexample}{#1}%
}

\newcommand\@newparaexample@star[1]{%
	\@newexample@abstract{Exemple}{#1}%
}


% Change log
\newcommand\topic{\@ifstar{\@topic@star}{\@topic@no@star}}

\newcommand\@topic@no@star[1]{%
	\textbf{\textsc{#1}.}%
}

\newcommand\@topic@star[1]{%
	\textbf{\textsc{#1} :}%
}

\makeatother % Technical doc - END


\usepackage{tnscom}


\begin{document}

\renewcommand\labelitemi{\raisebox{0.125em}{\tiny\textbullet}}
\renewcommand{\labelitemii}{---}

\title{  %
	Le package \texttt{tnscom}:\\%
	outils communs à la suite \texttt{tns}\\%
	{\footnotesize Code source disponible sur \url{https://github.com/typensee-latex/tnscom.git}.}\\%
{\footnotesize Version \texttt{0.1.0-beta} développée et testée sur \macosxname{}.}%
}
\author{Christophe BAL}
\date{2020-07-15}{{{{{{{{{{{

\maketitle


\vspace{2em}

\hrule

\tableofcontents

\vspace{1.5em}

\hrule

\newpage

\section{Introduction}

Ce package contient quelques macros utiles à différents packages de la série \verb+tns+ qui a pour modeste ambition de rendre un plus sémantique l'écriture de document avec \LaTeX.
Une partie des macros sont pour les coulisses et d'autres directement pour de la mise en forme commune à différents packages.


% ---------------------- %


\section{A propos des macros standards redéfinies}

Certaines macros comme \verb+\frac+ sont un peu revues par \verb+tnscom+.
Dans ce cas, les versions standard restent accessibles en utilisant le préfixe \prefix{std} ce qui donne ici la macro \macro{stdfrac}.
\section{Deux séparateurs d'arguments par défaut}

La macro \macro{tnsmathsep} définit le séparateur d'arguments de premier niveau et \macro{tnsmathsubsep} celui des arguments de deuxième niveau.
Cette documentation utilisant l'option \verb+french+ de \verb+babel+, la valeur de 
\macro{tnsmathsep} est \fbox{\,\tnsmathsep$\vphantom{F}$\,} 
et celle de
\macro{tnsmathsubsep} est \fbox{\,\tnsmathsubsep$\vphantom{F}$\,} .
Sans ce choix, les valeurs de \macro{tnsmathsep} et \macro{tnsmathsubsep} seront \fbox{\,\tnsmathsubsep$\vphantom{F}$\,} et \fbox{\,\tnsmathsep$\vphantom{F}$\,} respectivement à l'anglaise.
\section{Quelques modifications générales}

\subsection{Espace et point-virgule avec l'option \texttt{french} de \texttt{babel}}

\textbf{Seulement si vous utilisez \texttt{babel} avec l'option \texttt{french}}, comme c'est le cas dans cette documentation alors vous verrez le même espacement autour du point-virgule en mode mathématique comme dans $A(x;y)$. Que c'est beau !
%\section{Quelques modifications générales}

\subsection{Espace et fractions}

Quand on utilise \macro{frac} ou \macro{dfrac} de petits espaces sont automatiquement ajoutés pour éviter d'avoir des traits de fraction trop petits. Le comportement par  défaut se retrouve en utilisant les macros \macro{stdfrac} et \macro{stddfrac}. Voici un exemple.

\begin{latexex}
$\frac{2}{3}  = \stdfrac{2}{3}$  ,
$\dfrac{2}{3} = \stddfrac{2}{3}$
\end{latexex}


% ---------------------- %
%\section{Quelques modifications générales}

\subsection{Espace et racines n-ièmes d'un réel}

\macro{sqrt} a été redéfinie pour ajouter un peu d'espaces sur la droite sous le radical avec aussi un meilleur placement de l'exposant pour l'éloigner du radical. Le comportement par défaut se retrouve en utilisant la macro \macro{stdsqrt}. Voici ce que cela donne.


\begin{latexex}
$\sqrt{2}   = \stdsqrt{2}$   ,
$\sqrt{x_2} = \stdsqrt{x_2}$

$\sqrt[n]{45} = \stdsqrt[n]{45}$ ,
$\sqrt[p]{7}  = \stdsqrt[p]{7}$
\end{latexex}


% ---------------------- %
%\section{Quelques modifications générales}

\subsection{Une variable \og symbolique \fg{}}

%Alignement test : $\symvar*$$\symvar$

Dans certains contextes, comme celui des opérateurs fonctionnels, il peut être utile d'indiquer un argument symboliquement via $\symvar$ par exemple sans faire référence précisément à une ou des variables nommées.
\prefix{symvar} est pour \whyprefix{sym}{bolic} \whyprefix{var}{iable} soit \inenglish{variable symbolique}.
Voici un exemple d'utilisation.

\begin{latexex}
 $| \symvar |$ ou
 $\frac{d\symvar}{dx}
= \frac{d}{dx}\symvar$
\end{latexex}


Si besoin, vous disposez d'un autre symbole via la version étoilée de \macro{symvar} pour obtenir $\symvar*$ comme ci-dessous.

\begin{latexex}
 $| \symvar* |$ ou
 $\frac{d\symvar*}{dx}
= \frac{d}{dx}\symvar*$
\end{latexex}

\begin{remark}
	Comme $\symvar$ et $\symvar*$ sont des caractères, il faudra si besoin gérer les espaces autour via des \macro{kern} pour obtenir des choses comme  $|\kern.2ex\symvar\kern.2ex|$ et  $d\kern.25ex\symvar*$ par exemple.
\end{remark}



% ---------------------- %
\section{En coulisse...}

Cette section présente les macros dites \emph{\og privées \fg} qui sont proposées par \verb+tnscom+.
Toutes ces macros ont des noms commençant par \verb+\tns@+ et aucune version étoilée n'est proposée
\footnote{
	\og Explicite \fg{} est mieux que \og implicite \fg{}.
}.


\subsection{Nouvelle macro modifiant les arguments en amont de l'application d'une ancienne macro}

Comme cette fonctionnalité est utilisée par plusieurs packages de la suite \verb+tns+, une mini macro permet de faciliter la définition de ce type de nouvelle macro.
Par exemple, ci-dessous la 2\ieme{} macro ajoute juste une mise en forme particulière aux deux arguments juste avant d'appliquer la 1\iere{} macro.

\makeatletter
\newcommand\twoargs[2]{#1, #2}
\newcommand\modify[1]{\textbf{([#1])}}
\def\newtwoargs{\tns@apply@macro@two@args\twoargs\modify}
\makeatother

\begin{latexex}
\twoargs{A}{B} : \newtwoargs{A}{B}
\end{latexex}


Ceci est géré facilement via la macro \macro{tns@apply@macro@two@args} comme suit en fournissant comme 1\ier{} argument la macro cible et pour 2\ieme{} celle qui va modifier en amont les arguments.

\begin{latexex-alone}
\newcommand\twoargs[2]{#1, #2}
\newcommand\modify[1]{\textbf{([#1])}}

\def\newtwoargs{\tns@apply@macro@two@args\twoargs\modify}
\end{latexex-alone}


\begin{remark}
	En interne la macro \macro{tns@apply@macro@two@args} est définie avec quatre arguments.
	En fait ci-dessus nous utilisons la machinerie \LaTeX{} qui va manger les deux arguments manquants lors de l'utilisation de \macro{newtwoargs}.
\end{remark}


% ---------------------- %
%\section{En coulisse...}

\subsection{Macro avec un \og multi-argument \fg}

La suite \verb+tns+ propose la possibilité d'avoir des macros avec un nombre variables d'arguments. Pour ce type de macros, le choix a été fait de passer via unique argument au sens \LaTeX{} mais contenant des \emph{\og sous-arguemnts \fg} séparés par des barres verticales \emph{(on dit \emph{\og pipe \fg} en anglais)}.
Nous parlerons de  \og multi-argument \fg.

\medskip

Voici un exemple d'utilisation possible \emph{(la fonctionnalité ci-dessous est en fait disponible via la macro \macro{coord} du package \texttt{tnsgeo} disponible sur \url{https://github.com/typensee-latex/tnsgeo.git})}.

\makeatletter
\newcommand\verticalcoord[1]{%
    \tns@multi@args{|}{#1}%
                   {\begin{bmatrix}}%
                   {\\}%
                   {\end{bmatrix}}%
}
\makeatother

\begin{latexex}
$\verticalcoord{1}$                 ,
$\verticalcoord{1 | 2}$             ,
$\verticalcoord{1 | 2 | 3}$         ,
$\verticalcoord{1 | 2 | 3 | 4}$     ,
$\verticalcoord{1 | 2 | 3 | 4 | 5}$ ...
\end{latexex}


La macro \macro{verticalcoord} a été définie comme suit via la macro privée \macro{tns@multi@args}. L'environnement \env{bmatrix} vient du très pratique package \verb+nicematrix+ qu'utilise déjà \verb+tnscom+.

\begin{latexex-alone}
\newcommand\verticalcoord[1]{%
    \tns@multi@args{|}{#1}           % Séparateur et argument au sens LaTeX
                   {\begin{bmatrix}} % Matériel avant
                   {\\}              % Ce qui remplace le séparateur
                   {\end{bmatrix}}   % Matériel après
}
\end{latexex-alone}


% ---------------------- %
%\section{En coulisse...}

\subsection{Intervalles \og généralisés \fg}

Il est possible définir facilement des sortes d'intervalles \og généralisés \fg{}.


% ---------------------- %


\newparaexample{Mode extensible}

Il est assez facile de définir une macro ayant le comportement suivant.

\makeatletter
\newcommand\strangeset[2]{%
    \tns@generic@interval@ext{\{} % 1e délimiteur
                             {#1} % 1e élément
                             {::} % Séparateur entre les deux éléments 
                             {#2} % 2e élément
                             {)}  % 2e délimiteur
}
\makeatother

\begin{latexex}
$\strangeset{a}{\dfrac{b}{c}}$
\end{latexex}


La macro \macro{strangeset} a été définie comme suit via \macro{tns@generic@interval@ext}.

\begin{latexex-alone}
\newcommand\strangeset[2]{%
    \tns@generic@interval@ext{\{} % 1e délimiteur
                             {#1} % 1e élément
                             {::} % Séparateur entre les deux éléments 
                             {#2} % 2e élément
                             {)}  % 2e délimiteur
}
\end{latexex-alone}


% ---------------------- %


\newparaexample{Mode semi-extensible}

Le mode semi-extensible correspond à des délimiteurs un peu plus grand qu'en mode non extensible comme le montre l'exemple ci-après.

\makeatletter
\newcommand\myinter[2]{%
    \tns@generic@interval@semi@ext{\{}%
                                  {#1}{::}{#2}%
                                  {)}%
}
\makeatother

\begin{latexex}
$\myinter{a}{\dfrac{b}{c}}$ ou
$\{ a :: \dfrac{b}{c} )$
\end{latexex}


Il suffit d'utiliser \macro{tns@generic@interval@semi@ext} au lieu de \macro{tns@generic@interval@ext}. Voici le code utilisé.

\begin{latexex-alone}
\newcommand\myinter[2]{%
    \tns@generic@interval@semi@ext{\{}%
                                  {#1}{::}{#2}%
                                  {)}%
}
\end{latexex-alone}


% ---------------------- %
%\section{En coulisse...}

\subsection{Décorer un opérateur}

Il est facile d'obtenir l'effet suivant via la macro \macro{tns@over@math@symbol}.

\makeatletter
\newcommand\eqtxt{\tns@over@math@symbol{texte}{=}}
\makeatother

\begin{latexex}
$1 \eqtxt un$
\end{latexex}


Le code qui a été utilisé est le suivant où l'on fournit en 1\ier{} le texte décoratif et en 2\ieme{} l'opérateur.


\begin{latexex-alone}
\newcommand\eqtxt{\tns@over@math@symbol{texte}{=}}
\end{latexex-alone}


% ---------------------- %
%\section{En coulisse...}

\subsection{Produit en croix}

Comme ce type de calculs apparait très souvent dans divers domaines, une macro privée est chargée de ce tout petit travail de mise en forme.

\begin{latexex}
\makeatletter
$\tns@prop@prod{\cdot} % Opérateur
               {1}{2}  % Les éléments
               {3}{4}$ % du tableau
\makeatother
\end{latexex}


% ---------------------- %
%\section{En coulisse...}

\subsection{Déterminant 2D ou produit en croix décoré}

Des macros privées permettent, suivant le contexte, d'écrire un calcul de proportionnalité ou bien celui d'un déterminant avec différentes mises en forme possibles.


% ---------------------- %


\newparaexample{La totale}

La macro \macro{tns@det@plane@deco} est facile d'utilisation comme le montre l'exemple suivant.

\begin{latexex}
\makeatletter
$\tns@det@plane@deco%
     {vec}     % Vecteurs visibles
     {u}       % 1e vecteur 2D
     {x}{y}    % Coord. du 1e vecteur 
     {v}       % 2e vecteur 2D
     {x'}{y'}$ % Coord. du 2e vecteur
\makeatother
\end{latexex}


% ---------------------- %


\newparaexample{Décoration mais sans vecteur}

En choisissant \verb+novec+ au lieu de \verb+vec+, les vecteurs ne seront pas affichés.

\begin{latexex}
\makeatletter
$\tns@det@plane@deco{novec}   %
                    {u}{x}{y} %
                    {v}{x'}{y'}$
\makeatother
\end{latexex}


% ---------------------- %


\newparaexample{Sans décoration mais avec les vecteurs}

En utilisant \macro{tns@det@plane@no@deco}, la boucle fléchée ne sera pas imprimée. Voici une 1\iere{} utilisation possible.

\begin{latexex}
\makeatletter
$\tns@det@plane@no@deco{vec}     %
                       {u}{x}{y} %
                       {v}{x'}{y'}$
\makeatother
\end{latexex}


% ---------------------- %


\newparaexample{Sans décoration ni vecteur}

\begin{latexex}
\makeatletter
$\tns@det@plane@no@deco{novec}   %
                       {u}{x}{y} %
                       {v}{x'}{y'}$
\makeatother
\end{latexex}


% ---------------------- %
\newpage

\section{Historique}

Nous ne donnons ici qu'un très bref historique récent
\footnote{
	On ne va pas au-delà de un an depuis la dernière version.
}
de \verb+tnscom+ à destination de l'utilisateur principalement.
Tous les changements sont disponibles uniquement en anglais dans le dossier \verb+change-log+ : voir le code source de \verb+tnscom+ sur \verb+github+.

\begin{description}
% Changes shown - START

    \medskip
    \item[2020-07-15] Nouvelle version mineure \verb+0.1.0-beta+.
    
    \begin{itemize}[itemsep=.5em]
        \item \topic*{Symboles} les nouvelles macros \macro{symvar} et \macro{symvar*} produisent un disque plein et un carré plein permettant par exemple d'indiquer symboliquement une ou des variables.
        
        \separation
    \end{itemize}
% ------------------------ %

    \medskip
    \item[2020-07-10] Première version \verb+0.0.0-beta+.
% ------------------------ %

% Changes shown - END 
\end{description}


\newpage
\section{Toutes les fiches techniques} \label{techincal-ids}


\subsection{Deux séparateurs d'arguments par défaut}


\IDmacro*{tnsmathsep}{0}

\IDmacro*{tnsmathsubsep}{0}

\subsection{Quelques modifications générales}

\subsubsection{Espace et fractions}


\IDmacro*{frac    }{2}

\IDmacro*{dfrac   }{2}

\extraspace

\IDmacro*{stdfrac }{2}

\IDmacro*{stddfrac}{2}


\IDarg{1} le numérateur.

\IDarg{2} le dénominateur.
\subsubsection{Espace et racines n-ièmes d'un réel}


\IDmacro{sqrt   }{1}{1}

\IDmacro{stdsqrt}{1}{1}


\IDoption{} l'exposant à indiquer pour une racine n-ième.

\IDarg{} le radicande, c'est à dire ce qui sera écrit sous le radical.
\subsubsection{Une variable \og symbolique \fg{}}


\IDmacro*{symvar}{0}

\IDmacro*{symvar*}{0}
\subsection{En coulisse...}

\subsubsection{Nouvelle macro modifiant les arguments en amont de l'application d'une ancienne macro}


\IDmacro*{tns@apply@macro@two@args}{4}

\IDarg{1} macro finale avec deux arguments et c'est tout.

\IDarg{2} macro qui sera appliqué sur chacun des deux arguments avant appel la macro donné en argument 1.

\IDarg{$\!$s 3-4} ils sont là pour la définition abstraite mais en pratique ils ne seront pas utilisés
Ils correspondent aux arguments de la nouvelle macro fournie par \macro{tns@apply@macro@two@args}.
\subsubsection{Macro avec un \og multi-argument \fg}


\IDmacro*{tns@multi@args}{5}

\IDarg{1} le séparateur de \emph{\og sous-arguemnts \fg}.

\IDarg{2} le \og multi-argument \fg.

\IDarg{2} le matériel a ajouté avant.

\IDarg{4} le matériel que l'on met à la place du séparateur donné en 1\ier{} argument.

\IDarg{5} le matériel a ajouté après.
\subsubsection{Intervalles \og généralisés \fg}


\IDmacro*{tns@generic@interval@ext     }{5}

\IDmacro*{tns@generic@interval@semi@ext}{5}


\IDarg{1} le 1\ier{} délimiteur qui est à gauche.

\IDarg{2} le 1\ier{} élément de l'intervalle \og généralisé \fg.

\IDarg{3} le séparateur entre les deux éléments.

\IDarg{4} le 2\ieme{} élément de l'intervalle \og généralisé \fg.

\IDarg{5} le 2\ier{} délimiteur qui est à droite.
\subsubsection{Décorer un opérateur}


\IDmacro*{tns@over@math@symbol}{2}

\IDarg{1} le texte à ajouter au-dessus.

\IDarg{2} le symbole à décorer.
\subsubsection{Produit en croix}


\IDmacro*{tns@prop@prod}{5}


\IDarg{1} l'opérateur de multiplication à imprimer.

\IDarg{$\!$s 2-5} les $4$ éléments du tableau de proportionalité lus ligne par ligne.
\subsubsection{Déterminant 2D ou produit en croix décoré}


\IDmacro*{tns@det@plane@deco}{7}

\IDmacro*{tns@det@plane@no@deco}{7}


\IDarg{1} \verb+vec+ ou \verb+novec+ suivant que l'on veut afficher ou non les vecteurs. 

\IDarg{2} le 1\ier{} vecteur.

\IDarg{\!s 3-4} les coordonnées du 1\ier{} vecteur.

\IDarg{5} le 2\ieme{} vecteur.

\IDarg{\!s 6-7} les coordonnées du 2\ieme{} vecteur.


\end{document}
