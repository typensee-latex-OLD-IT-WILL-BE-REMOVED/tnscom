\documentclass[12pt,a4paper]{article}

\usepackage[french]{babel}

\makeatletter
	\input{../config/header[fr].sty}
	
	% == PACKAGES USED == %

\RequirePackage{siunitx}


% == DEFINITIONS == %

\@ifpackagewith{babel}{french}{
    \sisetup{
        list-final-separator  = {et},
        list-pair-separator   = {et},
        range-phrase          = {à},
        input-product         = {*},
        output-decimal-marker = {,},
        group-minimum-digits  = 5
    }
}{}
\makeatother


\begin{document}

\section{Travailler avec des unités S.I.}

L'excellent package \verb#siunitx# étant chargé par \verb+tnscom+, il devient facile d'obtenir les choses suivantes de façon sémantique.
Indiquons que \textbf{les conventions d'écriture sont françaises dès lors que vous aurez chargé \texttt{babel} avec l'option \texttt{french}}
\footnote{
	Ceci est un petit réglage simple à ajouter mais que \texttt{siunitx} ne propose par défaut. C'est un peu dommage.
}
comme c'est le cas pour cette documentation.
Ceci permet par exemple de taper \verb#3.6# pour obtenir \num{3.6} ci-dessous.

\begin{latexex}
$\ang{180} = \SI{\pi}{\radian}$

$\SI{1}{m.s^{-1}} = \SI{3.6}{km.h^{-1}}$
\end{latexex}


Dans l'exemple suivant, notez que les nombres à quatre chiffres comme  $\num{1234}$ sont écrits sans espace contrairement à ceux en ayant au moins cinq comme $\num{12345}$ car c'est l'usage typographique en France.
Notez aussi les facilités données via la saisie de \verb#*# et \verb#e# dans l'argument de \macro{num} \emph{(c'est comme si l'on tapait sur une calculatrice)},
ainsi que la possibilité d'utiliser des espaces pour améliorer la saisie des nombres dans le code \LaTeX.

\begin{latexex}
$\num{123*1000} = \num{123000}$

$\num{123e3} = \num{123 000}$
\end{latexex}


\begin{remark}
	Pour les curieux, les réglages utilisés pour le moment sont les suivants.
	
\begin{latexex-alone}
\sisetup{
    list-final-separator  = {et},
    list-pair-separator   = {et},
    range-phrase          = {à},
    input-product         = {*},
    output-decimal-marker = {,},
    group-minimum-digits  = 5
}
\end{latexex-alone}	
\end{remark}


\end{document}
