\documentclass[12pt,a4paper]{article}

\makeatletter
    \input{../config/header[fr].sty}

    % == PACKAGES USED == %

\RequirePackage{tikz}
\usetikzlibrary{calc}

\RequirePackage{nicematrix}
\RequirePackage{xstring}


% == DEFINITIONS == %

% Abstraction

%   #1 : & or && (no deco or deco)
%
%   #2 : 1st vector
%   #3 : 2nd vector

\newcommand\tns@det@plane@open@with@vect[3]{%
    \begin{vNiceMatrix}[first-row]
        {\scriptsize#2} #1 {\scriptsize#3} \\
}  


%   #1 : option
%   #2 : & or && (no deco or deco)
%
%   #3 : 1st vector
%   #4 : 2nd vector

\newcommand\tns@det@plane@open[4]{%
    \IfEqCase{#1}{
        {vec}{
            \tns@det@plane@open@with@vect{#2}{#3}{#4}
        }{novec}{
            \begin{vNiceMatrix}
        }%
    }[%
        \PackageError{tnscom}{unknown value for 1st argument}%
                             {you can use vec or novec.}%
	]
}        


% Entries
%
%   #1 : & or && (no deco or deco)
%
%   #2  #4        x  x'
%   #3  #5        y  y'

\newcommand\tns@det@plane@entries[5]{%
    #2 #1 #4 \\
    #3 #1 #5
}        


%   #1 : option for the formatting
%
%   #2  #5        u  v
%   
%   #3  #6        x  x'
%   #4  #7        y  y'

\newcommand\tns@det@plane@no@deco[7]{%
    \tns@det@plane@open{#1}{&}{#2}{#5}
        \tns@det@plane@entries{&}{#3}{#4}{#6}{#7}
    \end{vNiceMatrix}
}


\newcommand\tns@det@plane@deco[7]{%
    \tns@det@plane@open{#1}{&&}{#2}{#5}
        \tns@det@plane@entries{&&}{#3}{#4}{#6}{#7}
        %%
        \CodeAfter
        \begin{tikzpicture}
            \path (1-1.east) 
               -- node[below = 0.45em,
                       red,
                       circle,
                       inner sep = 1.25pt] 
                  (minus) {\tiny$-$}
                  (1-3.west);
            \draw [red,->]
                let \p1 = ($(minus.east) - (minus.center)$) in 
                    ([xshift = 0.5mm]1-1.east)
                    to[out = 0, in = 137.5]
                    (minus.50)
                    arc[start angle = 50,
                end angle = -230,
                radius = \x1] 
                    to[out = 42.5, in =190]
                    ([xshift = -0.3mm]1-1.east-|1-3.west);
        \end{tikzpicture}   
    \end{vNiceMatrix}
}



\makeatother



\begin{document}

%\section{En coulisse...}

\subsection{Déterminant 2D ou produits en croix décorés}

Des macros privées permettent, suivant le contexte, d'écrire un calcul de proportionnalité ou bien celui d'un déterminant avec différentes mises en forme possibles.


% ---------------------- %


\newparaexample{La totale en version courbée}

La macro \macro{tns@det@plane@deco} est facile d'utilisation comme le montre l'exemple suivant.

\begin{latexex}
\makeatletter
$\tns@det@plane@deco%
     {vec}     % Vecteurs visibles
     {u}       % 1e vecteur 2D
     {x}{y}    % Coord. du 1e vecteur 
     {v}       % 2e vecteur 2D
     {x'}{y'}  % Coord. du 2e vecteur
     {loop}$   % Une boucle explicative.
\makeatother
\end{latexex}


% ---------------------- %


\newparaexample{La totale en version produits en croix}

Un autre mise en forme avec les explications est disponible.

\begin{latexex}
\makeatletter
$\tns@det@plane@deco%
     {vec}     % Vecteurs visibles
     {u}       % 1e vecteur 2D
     {x}{y}    % Coord. du 1e vecteur 
     {v}       % 2e vecteur 2D
     {x'}{y'}  % Coord. du 2e vecteur
     {cross}$  % Une croix explicative.
\makeatother
\end{latexex}


% ---------------------- %


\newparaexample{Décoration mais sans vecteur}

En choisissant \verb+novec+ au lieu de \verb+vec+, les vecteurs ne seront pas affichés.

\begin{latexex}
\makeatletter
$\tns@det@plane@deco{novec}     %
                    {u}{x}{y}   %
                    {v}{x'}{y'} %
                    {loop}$
\makeatother
\end{latexex}


Ceci permet de produire un calcul de type produits en croix comme ci-dessous.

\begin{latexex}
\makeatletter
$\tns@det@plane@deco{novec}     %
                    {u}{x}{y}   %
                    {v}{x'}{y'} %
                    {cross}$
\makeatother
\end{latexex}

% ---------------------- %


\newparaexample{Sans décoration mais avec les vecteurs}

En utilisant \macro{tns@det@plane@no@deco}, la boucle fléchée ne sera pas imprimée. Voici une 1\iere{} utilisation possible.

\begin{latexex}
\makeatletter
$\tns@det@plane@no@deco{vec}     %
                       {u}{x}{y} %
                       {v}{x'}{y'}$
\makeatother
\end{latexex}


% ---------------------- %


\newparaexample{Sans décoration ni vecteur}

\begin{latexex}
\makeatletter
$\tns@det@plane@no@deco{novec}   %
                       {u}{x}{y} %
                       {v}{x'}{y'}$
\makeatother
\end{latexex}


% ---------------------- %


\subsection{Fiches techniques}

\IDmacro[a]{tns@det@plane@deco}{8}


\IDarg{1} \verb+vec+ ou \verb+novec+ suivant que l'on veut afficher ou non les vecteurs. 

\IDarg{2} le 1\ier{} vecteur.

\IDargs{3..4} les coordonnées du 1\ier{} vecteur.

\IDarg{5} le 2\ieme{} vecteur.

\IDargs{6..7} les coordonnées du 2\ieme{} vecteur.

\IDargs{8} \verb+loop+ ou \verb+cross+ suivant que l'on veut tracer une boucle ou une croix expliquant les calculs. 


\separation


\IDmacro[a]{tns@det@plane@no@deco}{7}


\IDarg{1} \verb+vec+ ou \verb+novec+ suivant que l'on veut afficher ou non les vecteurs. 

\IDarg{2} le 1\ier{} vecteur.

\IDargs{3..4} les coordonnées du 1\ier{} vecteur.

\IDarg{5} le 2\ieme{} vecteur.

\IDargs{6..7} les coordonnées du 2\ieme{} vecteur.

\end{document}