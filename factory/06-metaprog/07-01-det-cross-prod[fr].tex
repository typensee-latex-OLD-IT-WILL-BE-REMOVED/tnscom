\documentclass[12pt,a4paper]{article}

\makeatletter
    \input{../config/header[fr].sty}

    \input{07-det-cross-prod.sty}
\makeatother



\begin{document}

%\section{En coulisse...}

\subsection{Déterminant 2D et critère de colinéarité}

Des macros privées permettent, suivant le contexte, d'écrire le critère de colinéarité ou bien le calcul d'un déterminant 2D avec différentes mises en forme possibles.


% ---------------------- %


\newparaexample{La totale en version courbée}

La macro \macro{tns@det@plane@deco} est facile d'utilisation comme le montre l'exemple suivant.

\begin{latexex}
\makeatletter
$\tns@det@plane@deco@loop%
     {vec}     % Vecteurs visibles
     {u}       % 1e vecteur 2D
     {x}{y}    % Coord. du 1e vecteur 
     {v}       % 2e vecteur 2D
     {x'}{y'}$ % Coord. du 2e vecteur
\makeatother
\end{latexex}


% ---------------------- %


\newparaexample{La totale en version produits en croix}

Une autre mise en forme avec les explications est disponible.

\begin{latexex}
\makeatletter
$\tns@det@plane@deco@cross%
     {vec}     % Vecteurs visibles
     {u}       % 1e vecteur 2D
     {x}{y}    % Coord. du 1e vecteur 
     {v}       % 2e vecteur 2D
     {x'}{y'}$ % Coord. du 2e vecteur
\makeatother
\end{latexex}


% ---------------------- %


\newparaexample{Décoration mais sans vecteur}

En choisissant \verb+novec+ au lieu de \verb+vec+, les vecteurs ne seront pas affichés.

\begin{latexex}
\makeatletter
$\tns@det@plane@deco@loop{novec}    %
                         {u}{x}{y}  %
                         {v}{x'}{y'}$
\makeatother
\end{latexex}


Ceci permet de produire un calcul de type produits en croix comme ci-dessous.

\begin{latexex}
\makeatletter
$\tns@det@plane@deco@cross{novec}    %
                          {u}{x}{y}  %
                          {v}{x'}{y'}$
\makeatother
\end{latexex}

% ---------------------- %


\newparaexample{Sans décoration mais avec les vecteurs}

En utilisant \macro{tns@det@plane@no@deco}, la boucle fléchée ne sera pas imprimée. Voici une 1\iere{} utilisation possible.

\begin{latexex}
\makeatletter
$\tns@det@plane@no@deco{vec}      %
                       {u}{x}{y}  %
                       {v}{x'}{y'}$
\makeatother
\end{latexex}


% ---------------------- %


\newparaexample{Sans décoration ni vecteur}

\begin{latexex}
\makeatletter
$\tns@det@plane@no@deco{novec}    %
                       {u}{x}{y}  %
                       {v}{x'}{y'}$
\makeatother
\end{latexex}


% ---------------------- %


\subsection{Fiches techniques}

\IDmacro[a]{tns@det@plane@deco@loop}{7}

\IDmacro[a]{tns@det@plane@deco@cross}{7}

\IDmacro[a]{tns@det@plane@no@deco}{7}

\IDarg{1} \verb+vec+ ou \verb+novec+ suivant que l'on veut afficher ou non les vecteurs. 

\IDarg{2} le 1\ier{} vecteur.

\IDargs{3..4} les coordonnées du 1\ier{} vecteur.

\IDarg{5} le 2\ieme{} vecteur.

\IDargs{6..7} les coordonnées du 2\ieme{} vecteur.


\end{document}