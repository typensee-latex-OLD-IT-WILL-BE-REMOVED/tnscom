\documentclass[12pt,a4paper]{article}

\makeatletter
    \input{../config/header[fr].sty}

    \input{07-det-cross-prod.sty}
\makeatother



\begin{document}

%\section{En coulisse...}

\subsection{Produit vectoriel expliqué}

Une macro privée permet d'écrire un calcul de produit vectoriel
\footnote{
	Bien qu'utilisée juste par \texttt{tnsgeo}, cette fonctionnalité a été placée dans \texttt{tnscom} car son codage a beaucoup de points communs avec celui des déterminants 2D et du critère de colinéarité. Ceci facilite donc la maintenance.
}.


% ---------------------- %


\newparaexample{Avec des boucles explicatives}

La macro \macro{tns@cross@prod@deco} est facile d'utilisation comme le montre l'exemple suivant.

\begin{latexex}
\makeatletter
$\tns@cross@prod@deco@loop%
     {vec}         % Vecteurs visibles
     {u}           % 1e vecteur 2D
     {x}{y}{z}     % Coord. du 1e vecteur 
     {v}           % 2e vecteur 2D
     {x'}{y'}{z'}$ % Coord. du 2e vecteur
\makeatother
\end{latexex}


% ---------------------- %


\newparaexample{Avec des croix explicatives}

Une autre mise en forme avec les explications est disponible.

\begin{latexex}
\makeatletter
$\tns@cross@prod@deco@cross%
     {vec}          %
     {u}{x}{y}{z}   % 
     {v}{x'}{y'}{z'}$
\makeatother
\end{latexex}


% ---------------------- %


\newparaexample{Décoration mais sans vecteur}

En choisissant \verb+novec+ au lieu de \verb+vec+, les vecteurs ne seront pas affichés.

\begin{latexex}
\makeatletter
$\tns@cross@prod@deco@loop%
     {novec}        %
     {u}{x}{y}{z}   % 
     {v}{x'}{y'}{z'}$
ou
$\tns@cross@prod@deco@cross%
     {novec}        %
     {u}{x}{y}{z}   % 
     {v}{x'}{y'}{z'}$
\makeatother
\end{latexex}


% ---------------------- %


\newparaexample{Sans décoration mais avec les vecteurs}

En utilisant \macro{tns@cross@prod@no@deco}, la boucle fléchée ne sera pas imprimée. Voici une 1\iere{} utilisation possible.

\begin{latexex}
\makeatletter
$\tns@cross@prod@no@deco%
     {vec}          %
     {u}{x}{y}{z}   % 
     {v}{x'}{y'}{z'}$
\makeatother
\end{latexex}


% ---------------------- %


\newparaexample{Sans décoration ni vecteur}

\begin{latexex}
\makeatletter
$\tns@cross@prod@no@deco%
     {novec}        %
     {u}{x}{y}{z}   % 
     {v}{x'}{y'}{z'}$
\makeatother
\end{latexex}


% ---------------------- %


\subsection{Fiche technique}

\IDmacro[a]{tns@cross@prod@deco@loop }{9}

\IDmacro[a]{tns@cross@prod@deco@cross}{9}

\IDmacro[a]{tns@cross@prod@no@deco   }{9}

\IDarg{1} \verb+vec+ ou \verb+novec+ suivant que l'on veut afficher ou non les vecteurs. 

\IDarg{2} le 1\ier{} vecteur.

\IDarg{3..5} les coordonnées du 1\ier{} vecteur.

\IDarg{6} le 2\ieme{} vecteur.

\IDarg{7..9} les coordonnées du 2\ieme{} vecteur.

\end{document}
