\documentclass[12pt,a4paper]{article}

\makeatletter
    \usepackage[utf8]{inputenc}
\usepackage[T1]{fontenc}
\usepackage{ucs}

\usepackage[french]{babel,varioref}

\usepackage[top=2cm, bottom=2cm, left=1.5cm, right=1.5cm]{geometry}
\usepackage{enumitem}

\usepackage{multicol}

\usepackage{makecell}

\usepackage{color}
\usepackage{hyperref}
\hypersetup{
    colorlinks,
    citecolor=black,
    filecolor=black,
    linkcolor=black,
    urlcolor=black
}

\usepackage{amsthm}

\usepackage{tcolorbox}
\tcbuselibrary{listingsutf8}

\usepackage{ifplatform}

\usepackage{ifthen}

\usepackage{cbdevtool}


% Sections numbering

%\renewcommand\thechapter{\Roman{chapter}.}
\renewcommand\thesection{\arabic{section}.}
\renewcommand\thesubsection{\alph{subsection}.}
\renewcommand\thesubsubsection{\roman{subsubsection}.}


% MISC

\newtcblisting{latexex}{%
	sharp corners,%
	left=1mm, right=1mm,%
	bottom=1mm, top=1mm,%
	colupper=red!75!blue,%
	listing side text
}

\newtcbinputlisting{\inputlatexex}[2][]{%
	listing file={#2},%
	sharp corners,%
	left=1mm, right=1mm,%
	bottom=1mm, top=1mm,%
	colupper=red!75!blue,%
	listing side text
}


\newtcblisting{latexex-flat}{%
	sharp corners,%
	left=1mm, right=1mm,%
	bottom=1mm, top=1mm,%
	colupper=red!75!blue,%
}

\newtcbinputlisting{\inputlatexexflat}[2][]{%
	listing file={#2},%
	sharp corners,%
	left=1mm, right=1mm,%
	bottom=1mm, top=1mm,%
	colupper=red!75!blue,%
}


\newtcblisting{latexex-alone}{%
	sharp corners,%
	left=1mm, right=1mm,%
	bottom=1mm, top=1mm,%
	colupper=red!75!blue,%
	listing only
}

\newtcbinputlisting{\inputlatexexalone}[2][]{%
	listing file={#2},%
	sharp corners,%
	left=1mm, right=1mm,%
	bottom=1mm, top=1mm,%
	colupper=red!75!blue,%
	listing only
}


\newcommand\inputlatexexcodeafter[1]{%
	\begin{center}
		\input{#1}
	\end{center}

	\vspace{-.5em}
	
	Le rendu précédent a été obtenu via le code suivant.
	
	\inputlatexexalone{#1}
}


\newcommand\inputlatexexcodebefore[1]{%
	\inputlatexexalone{#1}
	\vspace{-.75em}
	\begin{center}
		\textit{\footnotesize Rendu du code précédent}
		
		\medskip
		
		\input{#1}
	\end{center}
}


\newcommand\env[1]{\texttt{#1}}
\newcommand\macro[1]{\env{\textbackslash{}#1}}



\setlength{\parindent}{0cm}
\setlist{noitemsep}

\theoremstyle{definition}
\newtheorem*{remark}{Remarque}

\usepackage[raggedright]{titlesec}

\titleformat{\paragraph}[hang]{\normalfont\normalsize\bfseries}{\theparagraph}{1em}{}
\titlespacing*{\paragraph}{0pt}{3.25ex plus 1ex minus .2ex}{0.5em}


\newcommand\separation{
	\medskip
	\hfill\rule{0.5\textwidth}{0.75pt}\hfill
	\medskip
}


\newcommand\extraspace{
	\vspace{0.25em}
}


\newcommand\whyprefix[2]{%
	\textbf{\prefix{#1}}-#2%
}

\newcommand\mwhyprefix[2]{%
	\texttt{#1 = #1-#2}%
}

\newcommand\prefix[1]{%
	\texttt{#1}%
}


\newcommand\inenglish{\@ifstar{\@inenglish@star}{\@inenglish@no@star}}

\newcommand\@inenglish@star[1]{%
	\emph{\og #1 \fg}%
}

\newcommand\@inenglish@no@star[1]{%
	\@inenglish@star{#1} en anglais%
}


\newcommand\ascii{\texttt{ASCII}}


% Example
\newcounter{paraexample}[subsubsection]

\newcommand\@newexample@abstract[2]{%
	\paragraph{%
		#1%
		\if\relax\detokenize{#2}\relax\else {} -- #2\fi%
	}%
}



\newcommand\newparaexample{\@ifstar{\@newparaexample@star}{\@newparaexample@no@star}}

\newcommand\@newparaexample@no@star[1]{%
	\refstepcounter{paraexample}%
	\@newexample@abstract{Exemple \theparaexample}{#1}%
}

\newcommand\@newparaexample@star[1]{%
	\@newexample@abstract{Exemple}{#1}%
}


% Change log
\newcommand\topic{\@ifstar{\@topic@star}{\@topic@no@star}}

\newcommand\@topic@no@star[1]{%
	\textbf{\textsc{#1}.}%
}

\newcommand\@topic@star[1]{%
	\textbf{\textsc{#1} :}%
}



    % == PACKAGES USED == %

\RequirePackage{relsize}


% == DEFINITIONS == %

% Math tools - Intervals and co

\newcommand\tns@extra@vphantom{%
    \vphantom{\relsize{1.25}{\text{$\displaystyle F_1^2$}}}%
}


% #1 : left symbol
% #2 : 1st part
% #3 : 2nd part
% #4 : 3rd part
% #5 : right symbol
\newcommand\tns@generic@interval@semi@ext[5]{%
    \ensuremath{%
        \left#1 \tns@extra@vphantom \right. \kern-.25em%
        #2 #3 #4%
        \left. \tns@extra@vphantom \kern-.05em \right#5%
    }%
}


% #1 : left symbol
% #2 : 1st part
% #3 : 2nd part
% #4 : 3rd part
% #5 : right symbol
\newcommand\tns@generic@interval@ext[5]{%
    \ensuremath{%
        \left#1 #2 #3 #4 \right#5%
    }%
}


\makeatother

\usepackage{amsmath}


\begin{document}

%\section{En coulisse...}

\subsection{Intervalles \og généralisés \fg}

Il est possible définir facilement des sortes d'intervalles \og généralisés \fg{}.


% ---------------------- %


\newparaexample{Mode extensible}

Il est assez facile de définir une macro ayant le comportement suivant.

\makeatletter
\newcommand\strangeset[2]{%
    \tns@generic@interval@ext{\{} % 1e délimiteur
                             {#1} % 1e élément
                             {::} % Séparateur entre les deux éléments 
                             {#2} % 2e élément
                             {)}  % 2e délimiteur
}
\makeatother

\begin{latexex}
$\strangeset{a}{\dfrac{b}{c}}$
\end{latexex}


La macro \macro{strangeset} a été définie comme suit via \macro{tns@generic@interval@ext}.

\begin{latexex-alone}
\newcommand\strangeset[2]{%
    \tns@generic@interval@ext{\{} % 1e délimiteur
                             {#1} % 1e élément
                             {::} % Séparateur entre les deux éléments 
                             {#2} % 2e élément
                             {)}  % 2e délimiteur
}
\end{latexex-alone}


% ---------------------- %


\newparaexample{Mode semi-extensible}

Le mode semi-extensible correspond à des délimiteurs un peu plus grand qu'en mode non extensible comme le montre l'exemple ci-après.

\makeatletter
\newcommand\myinter[2]{%
    \tns@generic@interval@semi@ext{\{}%
                                  {#1}{::}{#2}%
                                  {)}%
}
\makeatother

\begin{latexex}
$\myinter{a}{\dfrac{b}{c}}$ ou
$\{ a :: \dfrac{b}{c} )$
\end{latexex}


Il suffit d'utiliser \macro{tns@generic@interval@semi@ext} au lieu de \macro{tns@generic@interval@ext}. Voici le code utilisé.

\begin{latexex-alone}
\newcommand\myinter[2]{%
    \tns@generic@interval@semi@ext{\{}%
                                  {#1}{::}{#2}%
                                  {)}%
}
\end{latexex-alone}


% ---------------------- %


\subsection{Fiches techniques}

\IDmacro[a]{tns@generic@interval@ext     }{5}

\IDmacro[a]{tns@generic@interval@semi@ext}{5}


\IDarg{1} le 1\ier{} délimiteur qui est à gauche.

\IDarg{2} le 1\ier{} élément de l'intervalle \og généralisé \fg.

\IDarg{3} le séparateur entre les deux éléments.

\IDarg{4} le 2\ieme{} élément de l'intervalle \og généralisé \fg.

\IDarg{5} le 2\ier{} délimiteur qui est à droite.

\end{document}