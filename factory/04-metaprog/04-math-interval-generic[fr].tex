\documentclass[12pt,a4paper]{article}

\makeatletter
    \input{../config/header[fr].sty}

    \input{04-math-interval-generic.sty}
\makeatother

\usepackage{amsmath}


\begin{document}

%\section{En coulisse...}

\subsection{Intervalles \og généralisés \fg}

Il est possible définir facilement des sortes d'intervalles \og généralisés \fg{}.


% ---------------------- %


\newparaexample{Mode extensible}

Il est assez facile de définir une macro ayant le comportement suivant.

\makeatletter
\newcommand\strangeset[2]{%
    \tns@generic@interval@ext{\{} % 1e délimiteur
                             {#1} % 1e élément
                             {::} % Séparateur entre les deux éléments 
                             {#2} % 2e élément
                             {)}  % 2e délimiteur
}
\makeatother

\begin{latexex}
$\strangeset{a}{\dfrac{b}{c}}$
\end{latexex}


La macro \macro{strangeset} a été définie comme suit via \macro{tns@generic@interval@ext}.

\begin{latexex-alone}
\newcommand\strangeset[2]{%
    \tns@generic@interval@ext{\{} % 1e délimiteur
                             {#1} % 1e élément
                             {::} % Séparateur entre les deux éléments 
                             {#2} % 2e élément
                             {)}  % 2e délimiteur
}
\end{latexex-alone}


% ---------------------- %


\newparaexample{Mode semi-extensible}

Le mode semi-extensible correspond à des délimiteurs un peu plus grand qu'en mode non extensible comme le montre l'exemple ci-après.

\makeatletter
\newcommand\myinter[2]{%
    \tns@generic@interval@semi@ext{\{}%
                                  {#1}{::}{#2}%
                                  {)}%
}
\makeatother

\begin{latexex}
$\myinter{a}{\dfrac{b}{c}}$ ou
$\{ a :: \dfrac{b}{c} )$
\end{latexex}


Il suffit d'utiliser \macro{tns@generic@interval@semi@ext} au lieu de \macro{tns@generic@interval@ext}. Voici le code utilisé.

\begin{latexex-alone}
\newcommand\myinter[2]{%
    \tns@generic@interval@semi@ext{\{}%
                                  {#1}{::}{#2}%
                                  {)}%
}
\end{latexex-alone}


% ---------------------- %


\subsection{Fiches techniques}

\IDmacro[a]{tns@generic@interval@ext     }{5}

\IDmacro[a]{tns@generic@interval@semi@ext}{5}


\IDarg{1} le 1\ier{} délimiteur qui est à gauche.

\IDarg{2} le 1\ier{} élément de l'intervalle \og généralisé \fg.

\IDarg{3} le séparateur entre les deux éléments.

\IDarg{4} le 2\ieme{} élément de l'intervalle \og généralisé \fg.

\IDarg{5} le 2\ier{} délimiteur qui est à droite.

\end{document}