\documentclass[12pt,a4paper]{article}

\makeatletter
    \input{../config/header[fr].sty}

    % == PACKAGES USED == %

% == DEFINITIONS == %

% Meta tools - Apply same macro to all arguments

% #1      : main macro
% #2      : macro to apply to arguments
% #3 - #4 : the two arguments
\newcommand\tns@apply@macro@two@args[4]{%
    #1{#2{#3}}{#2{#4}}%
}


\makeatother



\begin{document}

\section{En coulisse...}

Cette section présente les macros dites \emph{\og privées \fg} qui sont proposées par \verb+tnscom+.
Toutes ces macros ont des noms commençant par \verb+\tns@+ et aucune version étoilée n'est proposée
\footnote{
	\og Explicite \fg{} est mieux que \og implicite \fg{}.
}.


\subsection{Nouvelle macro modifiant les arguments en amont de l'application d'une ancienne macro}

Comme cette fonctionnalité est utilisée par plusieurs packages de la suite \verb+tns+, une mini macro permet de faciliter la définition de ce type de nouvelle macro.
Par exemple, ci-dessous la 2\ieme{} macro ajoute juste une mise en forme particulière aux deux arguments juste avant d'appliquer la 1\iere{} macro.

\makeatletter
\newcommand\twoargs[2]{#1, #2}
\newcommand\modify[1]{\textbf{([#1])}}
\def\newtwoargs{\tns@apply@macro@two@args\twoargs\modify}
\makeatother

\begin{latexex}
\twoargs{A}{B} : \newtwoargs{A}{B}
\end{latexex}


Ceci est géré facilement via la macro \macro{tns@apply@macro@two@args} comme suit en fournissant comme 1\ier{} argument la macro cible et pour 2\ieme{} celle qui va modifier en amont les arguments.

\begin{latexex-alone}
\newcommand\twoargs[2]{#1, #2}
\newcommand\modify[1]{\textbf{([#1])}}

\def\newtwoargs{\tns@apply@macro@two@args\twoargs\modify}
\end{latexex-alone}


\begin{remark}
	En interne la macro \macro{tns@apply@macro@two@args} est définie avec quatre arguments.
	En fait ci-dessus nous utilisons la machinerie \LaTeX{} qui va manger les deux arguments manquants lors de l'utilisation de \macro{newtwoargs}.
\end{remark}


% ---------------------- %


\subsection{Fiches techniques}

\IDmacro[a]{tns@apply@macro@two@args}{4}

\IDarg{1} macro finale avec deux arguments et c'est tout.

\IDarg{2} macro qui sera appliqué sur chacun des deux arguments avant appel la macro donné en argument 1.

\IDargs{3..4} ils sont là pour la définition abstraite mais en pratique ils ne seront pas utilisés
Ils correspondent aux arguments de la nouvelle macro fournie par \macro{tns@apply@macro@two@args}.

\end{document}