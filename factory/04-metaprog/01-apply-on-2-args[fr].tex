\documentclass[12pt,a4paper]{article}

\makeatletter
    \usepackage[utf8]{inputenc}
\usepackage[T1]{fontenc}
\usepackage{ucs}

\usepackage[french]{babel,varioref}

\usepackage[top=2cm, bottom=2cm, left=1.5cm, right=1.5cm]{geometry}
\usepackage{enumitem}

\usepackage{multicol}

\usepackage{makecell}

\usepackage{color}
\usepackage{hyperref}
\hypersetup{
    colorlinks,
    citecolor=black,
    filecolor=black,
    linkcolor=black,
    urlcolor=black
}

\usepackage{amsthm}

\usepackage{tcolorbox}
\tcbuselibrary{listingsutf8}

\usepackage{ifplatform}

\usepackage{ifthen}

\usepackage{cbdevtool}


% Sections numbering

%\renewcommand\thechapter{\Roman{chapter}.}
\renewcommand\thesection{\arabic{section}.}
\renewcommand\thesubsection{\alph{subsection}.}
\renewcommand\thesubsubsection{\roman{subsubsection}.}


% MISC

\newtcblisting{latexex}{%
	sharp corners,%
	left=1mm, right=1mm,%
	bottom=1mm, top=1mm,%
	colupper=red!75!blue,%
	listing side text
}

\newtcbinputlisting{\inputlatexex}[2][]{%
	listing file={#2},%
	sharp corners,%
	left=1mm, right=1mm,%
	bottom=1mm, top=1mm,%
	colupper=red!75!blue,%
	listing side text
}


\newtcblisting{latexex-flat}{%
	sharp corners,%
	left=1mm, right=1mm,%
	bottom=1mm, top=1mm,%
	colupper=red!75!blue,%
}

\newtcbinputlisting{\inputlatexexflat}[2][]{%
	listing file={#2},%
	sharp corners,%
	left=1mm, right=1mm,%
	bottom=1mm, top=1mm,%
	colupper=red!75!blue,%
}


\newtcblisting{latexex-alone}{%
	sharp corners,%
	left=1mm, right=1mm,%
	bottom=1mm, top=1mm,%
	colupper=red!75!blue,%
	listing only
}

\newtcbinputlisting{\inputlatexexalone}[2][]{%
	listing file={#2},%
	sharp corners,%
	left=1mm, right=1mm,%
	bottom=1mm, top=1mm,%
	colupper=red!75!blue,%
	listing only
}


\newcommand\inputlatexexcodeafter[1]{%
	\begin{center}
		\input{#1}
	\end{center}

	\vspace{-.5em}
	
	Le rendu précédent a été obtenu via le code suivant.
	
	\inputlatexexalone{#1}
}


\newcommand\inputlatexexcodebefore[1]{%
	\inputlatexexalone{#1}
	\vspace{-.75em}
	\begin{center}
		\textit{\footnotesize Rendu du code précédent}
		
		\medskip
		
		\input{#1}
	\end{center}
}


\newcommand\env[1]{\texttt{#1}}
\newcommand\macro[1]{\env{\textbackslash{}#1}}



\setlength{\parindent}{0cm}
\setlist{noitemsep}

\theoremstyle{definition}
\newtheorem*{remark}{Remarque}

\usepackage[raggedright]{titlesec}

\titleformat{\paragraph}[hang]{\normalfont\normalsize\bfseries}{\theparagraph}{1em}{}
\titlespacing*{\paragraph}{0pt}{3.25ex plus 1ex minus .2ex}{0.5em}


\newcommand\separation{
	\medskip
	\hfill\rule{0.5\textwidth}{0.75pt}\hfill
	\medskip
}


\newcommand\extraspace{
	\vspace{0.25em}
}


\newcommand\whyprefix[2]{%
	\textbf{\prefix{#1}}-#2%
}

\newcommand\mwhyprefix[2]{%
	\texttt{#1 = #1-#2}%
}

\newcommand\prefix[1]{%
	\texttt{#1}%
}


\newcommand\inenglish{\@ifstar{\@inenglish@star}{\@inenglish@no@star}}

\newcommand\@inenglish@star[1]{%
	\emph{\og #1 \fg}%
}

\newcommand\@inenglish@no@star[1]{%
	\@inenglish@star{#1} en anglais%
}


\newcommand\ascii{\texttt{ASCII}}


% Example
\newcounter{paraexample}[subsubsection]

\newcommand\@newexample@abstract[2]{%
	\paragraph{%
		#1%
		\if\relax\detokenize{#2}\relax\else {} -- #2\fi%
	}%
}



\newcommand\newparaexample{\@ifstar{\@newparaexample@star}{\@newparaexample@no@star}}

\newcommand\@newparaexample@no@star[1]{%
	\refstepcounter{paraexample}%
	\@newexample@abstract{Exemple \theparaexample}{#1}%
}

\newcommand\@newparaexample@star[1]{%
	\@newexample@abstract{Exemple}{#1}%
}


% Change log
\newcommand\topic{\@ifstar{\@topic@star}{\@topic@no@star}}

\newcommand\@topic@no@star[1]{%
	\textbf{\textsc{#1}.}%
}

\newcommand\@topic@star[1]{%
	\textbf{\textsc{#1} :}%
}



    % == PACKAGES USED == %

% == DEFINITIONS == %

% #1      : main macro
% #2      : macro to apply to arguments
% #3 - #4 : the two arguments
\newcommand\tns@apply@macro@two@args[4]{%
    #1{#2{#3}}{#2{#4}}%
}


\makeatother



\begin{document}

\section{En coulisse...}

Cette section présente les macros dites \emph{\og privées \fg} qui sont proposées par \verb+tnscom+.
Toutes ces macros ont des noms commençant par \verb+\tns@+ et aucune version étoilée n'est proposée
\footnote{
	\og Explicite \fg{} est mieux que \og implicite \fg{}.
}.


\subsection{Nouvelle macro modifiant les arguments en amont de l'application d'une ancienne macro}

Comme cette fonctionnalité est utilisée par plusieurs packages de la suite \verb+tns+, une mini macro permet de faciliter la définition de ce type de nouvelle macro.
Par exemple, ci-dessous la 2\ieme{} macro ajoute juste une mise en forme particulière aux deux arguments juste avant d'appliquer la 1\iere{} macro.

\makeatletter
\newcommand\twoargs[2]{#1, #2}
\newcommand\modify[1]{\textbf{([#1])}}
\def\newtwoargs{\tns@apply@macro@two@args\twoargs\modify}
\makeatother

\begin{latexex}
\twoargs{A}{B} : \newtwoargs{A}{B}
\end{latexex}


Ceci est géré facilement via la macro \macro{tns@apply@macro@two@args} comme suit en fournissant comme 1\ier{} argument la macro cible et pour 2\ieme{} celle qui va modifier en amont les arguments.

\begin{latexex-alone}
\newcommand\twoargs[2]{#1, #2}
\newcommand\modify[1]{\textbf{([#1])}}

\def\newtwoargs{\tns@apply@macro@two@args\twoargs\modify}
\end{latexex-alone}


\begin{remark}
	En interne la macro \macro{tns@apply@macro@two@args} est définie avec quatre arguments.
	En fait ci-dessus nous utilisons la machinerie \LaTeX{} qui va manger les deux arguments manquants lors de l'utilisation de \macro{newtwoargs}.
\end{remark}


% ---------------------- %


\subsection{Fiches techniques}

\IDmacro*{tns@apply@macro@two@args}{4}

\IDarg{1} macro finale avec deux arguments et c'est tout.

\IDarg{2} macro qui sera appliqué sur chacun des deux arguments avant appel la macro donné en argument 1.

\IDarg{$\!$s 3-4} ils sont là pour la définition abstraite mais en pratique ils ne seront pas utilisés
Ils correspondent aux arguments de la nouvelle macro fournie par \macro{tns@apply@macro@two@args}.

\end{document}