\documentclass[12pt,a4paper]{article}

\makeatletter
    \input{../config/header[fr].sty}

    % == PACKAGES USED == %

\RequirePackage{nicematrix}

\RequirePackage{xparse}


% == DEFINITIONS == %

% Meta tools - Multi-arguments
%
% Source : the following lines come directly for the following post
%
%    * https://tex.stackexchange.com/a/475291/6880

\ExplSyntaxOn
% General purpose macro for defining other macros
%
% #1 : separator
% #2 : multiargument
% #3 : code before
% #4 : code between
% #5 : code after
    \NewDocumentCommand{\tns@multi@args}{mmmmmo}{
        \tnscom_multi@args:nnnnnn{#1}{#2}{#3}{#4}{#5}{#6}
    }
 
% Allocate a private variable
    \seq_new:N \l__tnscom_generic_seq

% The internal version of the general purpose macro
    \cs_new_protected:Nn \tnscom_multi@args:nnnnnn{
        % #1 : separator
        % #2 : multiargument
        % #3 : code before
        % #4 : code between
        % #5 : code after
        % #6 : ornament to items

        % A group allows nesting
        \group_begin:
            % Split the multi@argsument into parts
            \seq_set_split:Nnn \l__tnscom_generic_seq { #1 } { #2 }
            % Apply the ornament to the items
            \tl_if_novalue:nF { #6 }{
                \seq_set_eq:NN \l__tnscom_temp_seq \l__tnscom_generic_seq
                \seq_set_map:NNn \l__tnscom_generic_seq \l__tnscom_generic_seq { #6 }
            }
            % Execute the <code before>
            #3
            % Deliver the items, with the chosen material between them
            \seq_use:Nn \l__tnscom_generic_seq { #4 }
            % Execute the <code after>
            #5
            % End the group started at the beginning
        \group_end:
    }    
\ExplSyntaxOff



\makeatother

\usepackage{nicematrix}



\begin{document}

%\section{En coulisse...}

\subsection{Macro avec un \og multi-argument \fg{} à \og envelopper \fg}

La suite \verb+tns+ propose la possibilité d'avoir des macros avec un nombre variables d'arguments. Pour ce type de macros, le choix a été fait de passer via unique argument au sens \LaTeX{} mais contenant des \emph{\og sous-arguemnts \fg} séparés par des barres verticales \emph{(on dit \emph{\og pipe \fg} en anglais)}.
Nous parlerons de  \og multi-argument \fg.

\medskip

Voici un exemple d'utilisation possible \emph{(la fonctionnalité ci-dessous est en fait disponible via la macro \macro{coord} du package \texttt{tnsgeo} disponible sur \url{https://github.com/typensee-latex/tnsgeo.git})}.

\makeatletter
\newcommand\verticalcoord[1]{%
    \tns@multi@args{|}{#1}%
                   {\begin{bmatrix}}%
                   {\\}%
                   {\end{bmatrix}}%
}
\makeatother

\begin{latexex}
$\verticalcoord{1}$                 ,
$\verticalcoord{1 | 2}$             ,
$\verticalcoord{1 | 2 | 3}$         ,
$\verticalcoord{1 | 2 | 3 | 4}$     ,
$\verticalcoord{1 | 2 | 3 | 4 | 5}$ ...
\end{latexex}


La macro \macro{verticalcoord} a été définie comme suit via la macro privée \macro{tns@multi@args} en utilisant l'environnement \env{bmatrix} vient du très pratique package \verb+nicematrix+ qu'utilise déjà \verb+tnscom+.

\begin{latexex-alone}
\newcommand\verticalcoord[1]{%
    \tns@multi@args{|}{#1}           % Séparateur et argument au sens LaTeX
                   {\begin{bmatrix}} % Matériel avant
                   {\\}              % Ce qui remplace le séparateur
                   {\end{bmatrix}}   % Matériel après
}
\end{latexex-alone}


% ---------------------- %


\subsection{Fiches techniques}

\IDmacro[a]{tns@multi@args}{5}

\IDarg{1} le séparateur de \emph{\og sous-arguemnts \fg}.

\IDarg{2} le \og multi-argument \fg.

\IDarg{2} le matériel a ajouté avant.

\IDarg{4} le matériel que l'on met à la place du séparateur donné en 1\ier{} argument.

\IDarg{5} le matériel a ajouté après.

\end{document}