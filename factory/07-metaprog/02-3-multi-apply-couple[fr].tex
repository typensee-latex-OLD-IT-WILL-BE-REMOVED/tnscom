\documentclass[12pt,a4paper]{article}

\makeatletter
    \input{../config/header[fr].sty}

    % == PACKAGES USED == %

\RequirePackage{nicematrix}

\RequirePackage{xparse}


% == DEFINITIONS == %

% Meta tools - Multi-arguments
%
% Source : the following lines come directly for the following post
%
%    * https://tex.stackexchange.com/a/475291/6880

\ExplSyntaxOn
% General purpose macro for defining other macros
%
% #1 : separator
% #2 : multiargument
% #3 : code before
% #4 : code between
% #5 : code after
    \NewDocumentCommand{\tns@multi@args}{mmmmmo}{
        \tnscom_multi@args:nnnnnn{#1}{#2}{#3}{#4}{#5}{#6}
    }
 
% Allocate a private variable
    \seq_new:N \l__tnscom_generic_seq

% The internal version of the general purpose macro
    \cs_new_protected:Nn \tnscom_multi@args:nnnnnn{
        % #1 : separator
        % #2 : multiargument
        % #3 : code before
        % #4 : code between
        % #5 : code after
        % #6 : ornament to items

        % A group allows nesting
        \group_begin:
            % Split the multi@argsument into parts
            \seq_set_split:Nnn \l__tnscom_generic_seq { #1 } { #2 }
            % Apply the ornament to the items
            \tl_if_novalue:nF { #6 }{
                \seq_set_eq:NN \l__tnscom_temp_seq \l__tnscom_generic_seq
                \seq_set_map:NNn \l__tnscom_generic_seq \l__tnscom_generic_seq { #6 }
            }
            % Execute the <code before>
            #3
            % Deliver the items, with the chosen material between them
            \seq_use:Nn \l__tnscom_generic_seq { #4 }
            % Execute the <code after>
            #5
            % End the group started at the beginning
        \group_end:
    }    
\ExplSyntaxOff



\makeatother

\usepackage{nicematrix}



\begin{document}

%\section{En coulisse...}

\subsection{Appliquer une macro sur chaque couple de parties successives d'un \og multi-argument \fg}

\newparaexample{Appliquer partout}

Il est possible d'appliquer une macro sur chaque couple de parties successives partie d'un mutli-argument comme ci-dessous.


\makeatletter
\newcommand\decocouple[2]{%
    (#1)[#2]%
}

\newcommand\multiapplycouple[1]{%
    \tns@multi@apply@couple{\decocouple}{#1}
}
\makeatother


\begin{latexex}
\multiapplycouple{1|2|3}
\end{latexex}


La macro \macro{multiapplycouple} a été définie comme suit.

\begin{latexex-alone}
\newcommand\decocouple[2]{%
    (#1)[#2]%
}

\newcommand\multiapplycouple[1]{%
    \tns@multi@apply@couple{\decocouple}{#1}
}
\end{latexex-alone}


% ---------------------- %


\newparaexample{Choisir où appliquer}

Le compteur \verb#tns@multi@apply@couple@position# permet de connaître le numéro du couple à traiter.
Ceci permet de différencier par exemple le traitement du tout premier couple comme dans l'exemple suivant \emph{(c'est pour cela que le compteur a été créé)}. 


\makeatletter
\newcommand\decocoupleafterfirst[2]{%
    \ifnum\value{tns@multi@apply@couple@position}=1
        \{#1-#2\}%
    \else%
        (#1)[#2]%
    \fi
}

\newcommand\multiapplycoupleafterone[1]{%
    \tns@multi@apply@couple{\decocoupleafterfirst}{#1}
}
\makeatother

\begin{latexex}
\multiapplycoupleafterone{1|2|3}
\end{latexex}

La macro \macro{multiapplycoupleafterone} a été définie comme suit.

\begin{latexex-alone}
\newcommand\decocoupleafterfirst[2]{%
    \ifnum\value{tns@multi@apply@couple@position}=1
        \{#1-#2\}%
    \else%
        (#1)[#2]%
    \fi
}

\newcommand\multiapplycoupleafterone[1]{%
    \tns@multi@apply@couple{\decocoupleafterfirst}{#1}
}
\end{latexex-alone}


% ---------------------- %


\subsection{Fiches techniques}

\IDmacro[a]{tns@multi@apply@couple}{2}

\IDarg{1} la macro à appliquer à chaque couple de parties successives du \og multi-argument \fg.

\IDarg{2} le \og multi-argument \fg.

\end{document}
