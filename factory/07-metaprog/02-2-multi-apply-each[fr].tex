\documentclass[12pt,a4paper]{article}

\makeatletter
    \input{../config/header[fr].sty}

    % == PACKAGES USED == %

\RequirePackage{nicematrix}

\RequirePackage{xparse}


% == DEFINITIONS == %

% Meta tools - Multi-arguments
%
% Source : the following lines come directly for the following post
%
%    * https://tex.stackexchange.com/a/475291/6880

\ExplSyntaxOn
% General purpose macro for defining other macros
%
% #1 : separator
% #2 : multiargument
% #3 : code before
% #4 : code between
% #5 : code after
    \NewDocumentCommand{\tns@multi@args}{mmmmmo}{
        \tnscom_multi@args:nnnnnn{#1}{#2}{#3}{#4}{#5}{#6}
    }
 
% Allocate a private variable
    \seq_new:N \l__tnscom_generic_seq

% The internal version of the general purpose macro
    \cs_new_protected:Nn \tnscom_multi@args:nnnnnn{
        % #1 : separator
        % #2 : multiargument
        % #3 : code before
        % #4 : code between
        % #5 : code after
        % #6 : ornament to items

        % A group allows nesting
        \group_begin:
            % Split the multi@argsument into parts
            \seq_set_split:Nnn \l__tnscom_generic_seq { #1 } { #2 }
            % Apply the ornament to the items
            \tl_if_novalue:nF { #6 }{
                \seq_set_eq:NN \l__tnscom_temp_seq \l__tnscom_generic_seq
                \seq_set_map:NNn \l__tnscom_generic_seq \l__tnscom_generic_seq { #6 }
            }
            % Execute the <code before>
            #3
            % Deliver the items, with the chosen material between them
            \seq_use:Nn \l__tnscom_generic_seq { #4 }
            % Execute the <code after>
            #5
            % End the group started at the beginning
        \group_end:
    }    
\ExplSyntaxOff



\makeatother

\usepackage{nicematrix}



\begin{document}

%\section{En coulisse...}

\subsection{Appliquer une macro sur chaque partie d'un \og multi-argument \fg}

\newparaexample{Appliquer partout}

Il est possible d'appliquer une macro chaque partie d'un mutli-argument comme ci-dessous.

\makeatletter
\newcommand\decoone[2]{ ((#2)) }

\newcommand\multiapply[1]{%
    \tns@multi@apply@each{\decoone}{#1}
}
\makeatother

\begin{latexex}
\multiapply{1 | 2 | 3}
\end{latexex}

La macro \macro{multiapply} a été définie comme suit.
L'exemple \ref{tnscom-special-apply-first} un peu plus bas explique à quoi sert le 1\ier{} argument de \macro{decoone} non utilisé ici.

\begin{latexex-alone}
\newcommand\decoone[2]{ ((#2)) }

\newcommand\multiapply[1]{%
    \tns@multi@apply@each{\decoone}{#1}
}
\end{latexex-alone}


% ---------------------- %


\newparaexample{Choisir où appliquer} \label{tnscom-special-apply-first}

La macro \macro{tns@multi@apply@each} donne accès au numéro de l'argument à traiter via le 1\ier{} argument non utilisé dans l'exemple précédent.
Ceci rend possible de traiter à part le tout premier argument comme ci-dessous.

\makeatletter
\newcommand\decofirstandothers[2]{%
    \ifnum#1=1
        \fbox{#2} %
    \else%
        <<#2>> %
    \fi%
}

\newcommand\multiapplyafterone[1]{%
    \tns@multi@apply@each{\decofirstandothers}{#1} 
}
\makeatother

\begin{latexex}
\multiapplyafterone{1 | 2 | 3}
\end{latexex}

La macro \macro{multiapplyafterone} a été définie comme suit.

\begin{latexex-alone}
\newcommand\decofirstandothers[2]{%
    \ifnum#1=1
        \fbox{#2} %
    \else%
        <<#2>> %
    \fi%
}

\newcommand\multiapplyafterone[1]{%
    \tns@multi@apply@each{\decofirstandothers}{#1}
}
\end{latexex-alone}


% ---------------------- %


\subsection{Fiches techniques}

\IDmacro[a]{tns@multi@apply@each}{2}

\IDarg{1} la macro à appliquer à chaque partie du \og multi-argument \fg.
          Cette macro recevra deux arguments \emph{(le 1\ier{} est le numéro de la partie et le 2\ieme{} la partie elle-même)}.

\IDarg{2} le \og multi-argument \fg.

\end{document}