\documentclass[12pt,a4paper]{article}

\makeatletter
	\input{../config/header[fr].sty}

	\usepackage{01-symbolic-var}
\makeatother


\begin{document}

%\section{Quelques modifications générales}

\subsection{Une variable \og symbolique \fg{}}

%Alignement test : $\symvar*$$\symvar$

Dans certains contextes, comme celui des opérateurs fonctionnels, il peut être utile d'indiquer un argument symboliquement via $\symvar$ par exemple sans faire référence précisément à une ou des variables nommées.
Voici un exemple d'utilisation où \prefix{symvar} est pour \whyprefix{sym}{bolic} \whyprefix{var}{iable} soit \inenglish{variable symbolique}.

\begin{latexex}
 $| \symvar |$ ou
 $\frac{d\symvar}{dx}
= \frac{d}{dx} \symvar$
\end{latexex}


Si besoin, vous disposez d'autres symboles via l'argument optionnel de \macro{symvar} qui vaut \verb#1# par défaut.
Voici tous les symboles disponibles.

\begin{latexex}
$\symvar[1]$
$\symvar[2]$
$\symvar[3]$
\end{latexex}

\begin{remark}
	Comme les symboles sont juste des caractères au sens \LaTeX, il faudra si besoin gérer les espaces autour via par exemple des \macro{kern} pour obtenir des choses comme $|\kern.2ex\symvar\kern.2ex|$ et  $d\kern.25ex\symvar[2]$ qui ont été tapées \verb#$|\kern.2ex\symvar\kern.2ex|$# et \verb#$d\kern.25ex\symvar[2]$#.
\end{remark}


% ---------------------- %


\subsection{Fiches techniques}

\IDmacro[o]{symvar}{1}

\IDoption{} le numéro du symbole qui vaut \verb#1# par défaut.
\begin{itemize}
	\item \verb#1# donne $\symvar[1]$ .
	\item \verb#2# donne $\symvar[2]$ .
	\item \verb#3# donne $\symvar[3]$ .
%	\item \verb#4# donne $\symvar[4]$ .
\end{itemize}

\end{document}
